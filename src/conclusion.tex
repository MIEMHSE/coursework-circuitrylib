\documentclass[document.tex]{subfiles}
\begin{document}

\clearpage
\section{Выводы}
В ходе выполнения курсовой работы были изучены логические методы синтеза
комбинационных схем и произведены следующие действия по достижению поставленных
целей:

\begin{itemize}[noitemsep]
  \item Были изучены программные продукты, позволяющие упростить
  синтез комбинационных схем.
  \item Была обоснована разработка программного продукта для синтеза
  комбинационных логических схемотехнических устройств с возможностями
  отображения синтезируемых устройств в различных форматах.
  \item В результате реализации проекта, для синтеза схемотехнических устройств
  разработана библиотека CircuitryLib, которая позволяет автоматически
  синтезировать комбинационные схемы, а также является расширяемой благодаря
  изначально проведенной декомпозиции используемых сущностей.
\item Помимо библиотеки, был разработан интерфейс пользователя CircuitryLib-web,
  который позволяет произвести ограниченное методами по-умолчанию адаптеров
  изучение программной библиотеки.
\end{itemize}
 
Дальнейшие работы по проекту предполагают расширение количества синтезируемых
комбинационных схем (например, умножители), доработку библиотеки для
возможности задания базиса и ограничения на количество входов для
логических элементов, рассмотрение потенциально возможной интеграции с системой
симуляции схем по времени MyHDL. 

Общий прогресс разработки и само программное обеспечение можно получить из
системы контроля версий GitHub по следующим адресам:
\begin{itemize}[noitemsep]
  \item {\url{https://github.com/profitware/circuitrylib}} -- библиотека
  схемотехнического моделирования CircuitryLib;
  \item {\url{https://github.com/profitware/circuitrylib-web}} -- веб-интерфейс
  пользователя библиотеки.
\end{itemize}

\end{document}
