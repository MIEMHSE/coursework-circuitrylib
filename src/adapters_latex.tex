\documentclass[document.tex]{subfiles} 
\begin{document}

\clearpage
\subsection{Таблицы истинности в формате \LaTeX}
Адаптеры, отвечающие за генерацию таблиц истинности в формате \LaTeX, именуются
LatexTruthTableAdapter и DeviceMuxLatexTruthTableAdapter для
генерации простой таблицы истинности и таблицы истинности для мультиплексоров и
демультиплексоров, соответственно. Описание первого
представлено в листинге~\ref{lst:latex}.

\begin{listing}[ht]
\begin{minted}[linenos=true]{python}
class LatexTruthTableAdapter(AbstractAdapter):
    public_properties = ('latex_columns', 'latex_columns_names', 'latex_table')

    default_method = lambda self: \
        (r'\\begin{tabular}\{%(latex_columns)s}' +
         r'\\\hline\\%(latex_columns_names)s\\' +
         r'\hline\\%(latex_table)s\hline\\\end{tabular}') % \
        {'latex_columns': self.latex_columns,
         'latex_columns_names': self.latex_columns_names,
         'latex_table': self.latex_table}
\end{minted}
\caption{Программное описание класса адаптера \LaTeX}
\label{lst:latex}
\end{listing}

Описание адаптера для обработки мультиплексоров и демультиплексоров,
представленное в листинге~\ref{lst:muxlatex}, не сильно отличается от первого,
ибо только расширяет его функционал, не меняя сигнатур.
\begin{listing}[ht]
\begin{minted}[linenos=true]{python}
class DeviceMuxLatexTruthTableAdapter(LatexTruthTableAdapter):
    pass
\end{minted}
\caption{Программное описание класса адаптера \LaTeX для мультиплексоров}
\label{lst:muxlatex}
\end{listing}

\clearpage

Пример применения второго адаптера к устройству (в данном случае,
мультиплексор) хорошо иллюстрируется следующим кодом, представленным в
листинге~\ref{lst:muxlatexgen}.

\begin{listing}[ht]
\begin{minted}{pycon}
>>> from circuitry.adapters.latex.mux import DeviceMuxLatexTruthTableAdapter        
>>> from circuitry.devices.mux import DeviceMux                             
>>> 
>>> device_mux = DeviceMux(strobe_signals='v:1', address_signals='a:2',
...                        data_signals='d:4', output_signals='o:1',
...                        strobe_signals_subs=dict(v0=1),
...                        output_signals_subs=dict(o0=1))
>>> 
>>> print DeviceMuxLatexTruthTableAdapter(device_mux).default_method()
\begin{tabular}{|c|cc|cccc|c|}
\hline
$V_{0}$&$A_{0}$&$A_{1}$&$D_{0}$&$D_{1}$&$D_{2}$&$D_{3}$&$O_{0}$ \\
\hline 0&x&x&x&x&x&x&0 \\
1&0&0&1/0&x&x&x&1/0 \\
1&1&0&x&1/0&x&x&1/0 \\
1&0&1&x&x&1/0&x&1/0 \\
1&1&1&x&x&x&1/0&1/0 \\
\hline 
\end{tabular}
>>>
\end{minted}
\caption{Генерация кода \LaTeX}
\label{lst:muxlatexgen}
\end{listing}

Таблица истинности, сгенерированная адаптером, при включении в данный документ,
приведена в таблице~\ref{table:truth_table}.


\begin{table}[h]
\centering
\begin{tabular}{|c|cc|cccc|c|}
\hline
$V_{0}$&$A_{0}$&$A_{1}$&$D_{0}$&$D_{1}$&$D_{2}$&$D_{3}$&$O_{0}$ \\
\hline 0&x&x&x&x&x&x&0 \\
1&0&0&1/0&x&x&x&1/0 \\
1&1&0&x&1/0&x&x&1/0 \\
1&0&1&x&x&1/0&x&1/0 \\
1&1&1&x&x&x&1/0&1/0 \\
\hline 
\end{tabular}
\caption{Таблица истинности мультиплексора}
\label{table:truth_table}
\end{table}


\end{document}