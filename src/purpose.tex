\documentclass[document.tex]{subfiles} 
\begin{document}
\clearpage\section*{Введение}
\addcontentsline{toc}{section}{Введение}
В настоящее время существует некоторый спрос на релизацию решений по
автоматизации синтеза различного рода сущностей, ранее формировавшихся в ручном
или полуавтоматическом режимах. К таковым, например, можно отнести
логические схемы комбинационных схемотехнических устройств \cite{autogen}.

Промышленные стандарты и решения для синтеза таких устройств (например, VHDL и Verilog) не
позволяют в полной мере достичь гибкости в разработке и описании
схемотехнических устройств с помощью принципов объектно-ориентированного 
программирования и использования таких инструментов как графы и символьная 
алгебра логики. Невозможность разложить синтезированную схему на
отдельные логические вентили, влечет за собой
невозможность автоматизированной смены базиса и разрядности используемых
логических вентилей схем.

Эти недостатки не являются критичными, но увеличивают время 
синтеза нетиповых комбинационных схем по сравнению с потенциально возможной 
автоматической генерацией по параметрически заданным количествам входных и 
выходных портов с помощью объектно-ориентированного программирования. 

\clearpage\section{Цель работы}
Цель работы -- изучение логических методов синтеза комбинационных
схемотехнических устройств и разработка программного обеспечения для
автоматического синтеза по заданным параметрам входных и выходных сигналов для
представления устройств в различных форматах. 

Помимо разработки непосредственно библиотеки, пригодной для использования
программистами, необходимо разработать интерфейс пользователя, с помощью
которого можно задавать параметры разрядности входов и выходов устройств,
получая на выходе результаты преобразования в текстовом (таблицы истинности, код
MATLAB, код \LaTeX) и графическом (условно-графические обозначения, графы) виде.

\section{Обзор существующих решений}
Существуют аналогичные реализуемому в рамках курсовой работы решения, в том
числе на языке программирования Python. Из наиболее схожих по функционалу можно
выделить три решения -- MyHDL ({\url{http://www.myhdl.org/}}), Chips
({\url{http://dawsonjon.github.io/chips/}}) и BinPy
({\url{http://www.binpy.org/}}). Первые два делают упор на симуляции
схемотехнических устройств и позволяют генерировать VHDL и Verilog описания, но
не позволяют представить общую схему устройства в виде графов \cite{myhdldoc,
chipsdoc}.
Третье наиболее схоже с реализуемым проектом, но опять же не позволяет
производить декомпозицию синтезируемых устройств на отдельные логические
вентили, из-за чего возникает невозможность использования готовой кодовой базы
для реализации новых устройств с помощью наследования \cite{binpydoc}.
\end{document}