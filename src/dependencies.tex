\documentclass[document.tex]{subfiles} 
\begin{document}
\clearpage
\section{Внутренняя реализация программной библиотеки}

\subsection{Зависимости от внешних компонентов}
В рамках проекта было решено не разрабатывать базовые сущности (такие
как логические элементы символьной алгебры и графы) с нуля, а взять готовые
программные библиотеки со свободными лицензиями. Зависимости проекта на текущий
момент включают в себя следующие составные компоненты:
\begin{itemize}[noitemsep]
  \item SymPy ({\url{http://sympy.org/}}) -- библиотека для работы с
  символьной алгеброй, содержащая примитивы для описания логических
  выражений~\cite{sympydoc}; является базовой для синтеза схемотехнических
  устройств проекта.
  \item Pillow ({\url{http://python-imaging.github.io/}}) (поддерживаемый форк
  PIL) -- библиотека для работы с графическими изображениям~\cite{pillowdoc};
  используется для реализации адаптера вывода условно-графических обозначений устройств.
  \item NumPy ({\url{http://www.numpy.org/}}) -- библиотека для математических
  расчетов~\cite{numpydoc}; используется библиотекой SymPy.
  \item NetworkX ({\url{http://networkx.github.io/}}) -- библиотека для работы с
  графами~\cite{networkx}; используется в реализации адаптера отображения схем в
  виде графов и адаптера построения моделей Simulink.
  \item MatplotLib ({\url{http://matplotlib.org/}}) -- библиотека для
  визуального отображения связанных сущностей~\cite{matplotlib}; используется в
  реализации адаптера отображения схем в виде графов.
\end{itemize}

\clearpage\subsection{Генерация новых типов синтезируемых устройств}
Способ описания синезируемых устройств проекта базируется на нескольких
основных принципах:
\begin{itemize}[noitemsep]
  \item Класс описываемого устройства наследуется от абстрактного базового
  класса circuitry.devices.Device.
  \item Для устройства делается стандартное описание в формате PyDoc.
  \item Переопределяется кортеж mandatory{\_}signals, содержащий типы сигналов
  (типы strobe{\_}signals и output{\_}signals являются обязательными).
  \item Переопределяется кортеж mandatory{\_}signals{\_}using{\_}subs,
  содержащий типы сигналов, использующих подстановки (входы и выходы могут быть прямыми и
  инвертированными, тип strobe{\_}signals является обязательным).
  \item Переопределяется кортеж truth{\_}table{\_}signals, используемый всеми
  адаптерами генерации таблиц истинности и содержащий типы сигналов для
  включения в таблицы.
  \item Переопределяется словарь constraints, содержащий ключами наименования
  всех типов сигналов, перечисляемых в кортеже mandatory{\_}signals, а значениям
  словари с ключами min и max, определяющими ограничения по количеству для
  каждого из типов сигналов.
  \item В секции инициализации первым идет вызов конструктора родительского
  класса.
  \item Далее заполяется список функций functions, состоящий из комбинированных
  примитивов библиотеки символьной алгебры с именами сигналов или наборов
  других функций в качестве входных параметров.
  \item При необходимости автоматической генерации таблиц истинности для работы
  адаптеров, в конце секции инициализации вызывается функция
  {\_}generate{\_}through{\_}truth{\_}table с кортежом сигналов signals{\_}list,
  которые необходимо включить в сгенерированную таблицу истинности.
\end{itemize}

\clearpage\subsection{Генерация новых типов адаптеров}
Способ описания адаптеров для обработки синтезируемых устройств проекта
базируется на нескольких основных принципах:
\begin{itemize}[noitemsep]
  \item Класс адаптера наследуется от абстрактного базового класса
  circuitry.adapters.AbstractAdapter.
  \item Переопределяется кортеж public{\_}properties, содержащий
  наименования методов класса, доступных для вызова.
  \item Переопределяется функция default{\_}method, возвращающая результат
  обработки синтезируемого устройства с параметрами по-умолчанию (в случае
  графических изображений, возвращается имя файла с изображением).
  \item Переопределяется строка default{\_}content{\_}type, содержащая тип
  содержимого, возвращаемого методом по-умолчанию (text/plain для текста,
  image/png для графических изображений).
\end{itemize}

\end{document}