\documentclass[a4paper,14pt,oneside]{extarticle}

% Необходимые пакеты
\usepackage[T2A]{fontenc}
\usepackage{fancyvrb}
\usepackage{color}
\usepackage{minted}
\usepackage{etoolbox}

\AtBeginEnvironment{minted}{\singlespacing%
    \fontsize{12}{12}\selectfont}

\newminted{pycon}{bgcolor=bg, linenos=true, tabsize=4, fontsize=12pt}

\usepackage[utf8]{inputenc}
\usepackage[russian]{babel}
% \usepackage[pdftex,unicode]{hyperref}
\usepackage[warn]{mathtext}
\usepackage{amssymb,amsfonts,amsmath,cite,enumerate,float} % пакеты расширений
\usepackage{enumitem}
%\usepackage[dvips]{graphicx} % рисунки
\usepackage{subfiles}
\usepackage{mathtools}

\usepackage{hyperref}

\usepackage{graphicx}
\DeclareGraphicsExtensions{.png}

\usepackage{rotating,booktabs}

\usepackage{python}

% Python
% \newcommand{\logictable}[1]{%
% \begin{python}
% print 1
% \end{python}
% }

% Путь к рисункам
\graphicspath{{images/}} 

% Округление в большую сторону
\DeclarePairedDelimiter{\ceil}{\lceil}{\rceil}

% Меняем поля страницы
\usepackage{geometry}
\geometry{left=3cm} % левое поле
\geometry{right=1cm} % правое поле
\geometry{top=2cm} % верхнее поле
\geometry{bottom=2cm} % нижнее поле

\usepackage{indentfirst}

\usepackage{setspace}
\setlength{\parindent}{1cm}

\usepackage{epstopdf}


\usepackage{algorithm2e}

\linespread{1.3} % полуторный интервал
% \renewcommand{\rmdefault}{ftm} % Times New Roman

% \itemsep=0pt

\sloppy

% \makeatletter
% \renewcommand\section{\@startsection {section}{1}{\z@}%
%  {-3.5ex \@plus -1ex \@minus -.2ex}%
%  {2.3ex \@plus.2ex}%
%  {\centering\normalfont\Large\bfseries}}
% \makeatother

\makeatletter
\renewcommand{\@biblabel}[1]{#1.\hfil}
\makeatother
\addto\captionsrussian{\def\refname{Список использованных источников}}

\usepackage[tableposition=top]{caption}
\usepackage{subcaption}
\DeclareCaptionLabelFormat{gostfigure}{Рисунок #2}
\DeclareCaptionLabelFormat{gosttable}{Таблица #2}
\DeclareCaptionLabelFormat{gostlisting}{Листинг #2}
\DeclareCaptionLabelSeparator{gost}{~---~}
\captionsetup{labelsep=gost}
\captionsetup[figure]{labelformat=gostfigure}
\captionsetup[table]{labelformat=gosttable}
\captionsetup[listing]{labelformat=gostlisting}
\renewcommand{\thesubfigure}{\asbuk{subfigure}}

 
% Перечисления цифра.цифра
\renewcommand{\theenumi}{\arabic{enumi}}
\renewcommand{\labelenumi}{\arabic{enumi}}
\renewcommand{\theenumii}{.\arabic{enumii}}
\renewcommand{\labelenumii}{\arabic{enumi}.\arabic{enumii}}
\renewcommand{\theenumiii}{.\arabic{enumiii}}
\renewcommand{\labelenumiii}{\arabic{enumi}.\arabic{enumii}.\arabic{enumiii}}

% Нумерация разделов
\renewcommand{\thesection}{\arabic{section}}
\renewcommand{\thesubsection}{\arabic{section}.\arabic{subsection}}

\renewcommand{\figurename}{Рисунок}

% Наименование содержания
% \renewcommand\contentsname{Содержание}
% \def\contentsname{Содержание}

% Документ
\begin{document}

% Титульный лист
\begin{titlepage}
\newpage

\begin{center}
ФЕДЕРАЛЬНОЕ АГЕНТСТВО ПО ОБРАЗОВАНИЮ РФ \\
\vspace{1cm}
Московский институт электроники и математики \\*
Национального исследовательского университета \\*
Высшая Школа Экономики \\*
\hrulefill
\end{center}
 
\flushright{Кафедра Вычислительных систем и сетей}

\vspace{5em}

\begin{center}
\Large Курсовая работа \\ на тему:
\end{center}

\vspace{2.0em}
 
\begin{center}
\textsc{\textbf{Разработка программного обеспечения для моделирования \linebreak
цифровых комбинационных схемотехнических устройств}}
\end{center}

\begin{center}
по курсу \\ ``Схемотехника''
\end{center}

\vspace{5em}
 
\begin{flushleft}
Студент гр. СВ-71 \hrulefill Собко С.С. \\
\vspace{1.0em}
Преподаватель \hrulefill Черноусова Т.Г.
\end{flushleft}
 
\vspace{\fill}

\begin{center}
Москва 2014
\end{center}

\end{titlepage}

\subfile{chapters}

% Тело документа
\subfile{purpose} % Цель работы
\subfile{input} % Исходные данные
\subfile{description} % Принцип работы
\subfile{sourcedev_simple}
\subfile{sourcedev_mux}
\subfile{sourcedev_adder}
\subfile{sourcedev_cmp}
\subfile{adapters_matlab}
\subfile{adapters_graph}
\subfile{adapters_symbol}
\subfile{adapters_console}
\subfile{adapters_latex}
\subfile{dependencies} % Принцип работы
\subfile{conclusion}
\subfile{literature}
\end{document}