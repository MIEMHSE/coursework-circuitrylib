\documentclass[document.tex]{subfiles}
\begin{document}
\section{Синтез мультиплексора}
Заданный мультиплексор имеет N=14 входов на один выход, два прямых и один
обратный строб-сигнал. Таблица истинности устройства приведена в
таблице~\ref{tabular:logictable}. Число адресных сигналов m=4 ($A_{3}$, $A_{2}$,
$A_{1}$, $A_{0}$).
Из таблицы~\ref{tabular:logictable} видно, что $V_{0}$ и $V_{1}$ - прямые
строб-сигналы, а $V_{2}$ - обратный строб-сигнал. В случае, когда эти сигналы не
соответствуют норме, мультиплексор не работает - на его выходе сигнал
логического нуля.

При подаче нормальных строб-сигналов и нужного адресного кода, осуществляется
коммутация i-го входа мультиплексора с его выходом.

Логическое уравнение имеет вид:

\subfile{logicformula} % Таблица истинности

Для реализации этого мультиплексора необходимо иметь пять инверторов,
четырнадцать элементов 8И и один элемент 14ИЛИ.

\subfile{logictable} % Таблица истинности
\subfile{schema}


\end{document}
