\documentclass[document.tex]{subfiles}
\begin{document}

\section{Модель мультиплексора}
\subsection{О библиотеке}

При проектировании мультиплексора реализована программная библиотека для работы
со схемотехническими объектами CircuitryLib с использованием языка программирования
Python. Помимо генерации таблиц истинности и формул, условно-графического
изображения, она позволяет реализовать модели Simulink с использованием графов.
Модель, полученная после запуска программы представлена на рисунках ниже.

\begin{figure}[htb]
\begin{center}
\includegraphics[scale=1.0]{muxugo.png}
\caption{Условно-графическое изображение заданного мультиплексора}
\end{center}
\end{figure}

\begin{figure}[htb]
\begin{center}
\includegraphics[width=\linewidth]{simulink_pic1.png}
\caption{Модель Simulink -- newModel1}
\end{center}
\end{figure}

\begin{figure}[htb]
\begin{center}
\includegraphics[width=\linewidth]{simulink_pic2_1.png}
\caption{Модель Simulink -- newModel1/straight -- 1}
\end{center}
\end{figure}

\begin{figure}[htb]
\begin{center}
\includegraphics[width=\linewidth]{simulink_pic2_2.png}
\caption{Модель Simulink -- newModel1/straight -- 2}
\end{center}
\end{figure}

\begin{figure}[htb]
\begin{center}
\includegraphics[width=\linewidth]{simulink_pic2_3.png}
\caption{Модель Simulink -- newModel1/straight -- 3}
\end{center}
\end{figure}

\begin{figure}[htb]
\begin{center}
\includegraphics[width=\linewidth]{simulink_pic2_4.png}
\caption{Модель Simulink -- newModel1/straight -- 4}
\end{center}
\end{figure}

\end{document}
